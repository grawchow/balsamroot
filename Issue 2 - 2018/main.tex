% book example for classicthesis.sty
\documentclass[10pt,a6paper,footinclude=false,firstfoot=false,headinclude=true,open=any,DIV=6]{scrbook} % KOMA-Script book
\usepackage[paperheight=5.5in, paperwidth=4.25in]{geometry}
\usepackage[T1]{fontenc}
\usepackage{lipsum}
\usepackage[pdfspacing,manychapters]{zine}
\usepackage{cmzine}
\usepackage{amsthm}
\usepackage{ellipsis}
\usepackage{xcolor}
\usepackage{graphicx}
\usepackage{changepage}
\usepackage{wrapfig}
\usepackage{xfrac}
% \usepackage{poemscol}
\usepackage{verse}
\usepackage{enumitem}
% \usepackage{showframe}
% \usepackage{layouts}

%set font to anttor
    \usepackage[light,math]{anttor}

% set font to Calligra
    \usepackage{calligra}


    % \usepackage{tgpagella}

%sets font to Gentium
    \usepackage[T1]{fontenc}
    \usepackage{bookman}



\begin{document}

\thispagestyle{empty}
\hbox{}
\newpage
\pagenumbering{gobble}

%begin title page
    \setcounter{page}{1}
    % \newgeometry{left=.7cm, right=0cm, top=5cm}
    \newgeometry{left=1cm, right=1cm, top=5cm}
    \thispagestyle{empty}
    \pagecolor{geraniumTitleColor}


    % \hspace{-.3em}\inTitleCalligra{Fern}\hspace{-1.1em}\inTitleCaps{fern}

    % \hspace{1.6em}\inTitleDate{July \texttt{\^} August}

    % \hspace{-2em}\inTitleCalligra{Days}\hspace{-1.78em}\inTitleCaps{geranium}

    \begin{center}
    \inTitleCaps{balsam root}

    \noindent\inTitleDate{Issue II \texttt{\^} 2018}
    \end{center}

%end title pages
    \newpage
    \pagecolor{white}
    % \cleardoublepage
    \restoregeometry
    \normalfont
    \normalsize
    \color{black}

    \pagenumbering{gobble}
    \setcounter{page}{2}

\chapter{welcome to balsamroot}

Ah, Christmastime. A yule log burning in the fire, the latest issue of \spacedlowsmallcaps{balsamroot} nestled in under the tree, a warm brandy in your hands.

\vspace{1ex}

\tableofcontents

\newpage
% \thispagestyle{empty}
% \hbox{}
% \cleardoublepage
\pagenumbering{arabic}
\setcounter{page}{2}

\raggedbottom


\chapter{do you like hobbies?}

\spacedlowsmallcaps{What we're listening to}:


\vspace{2ex}
\titlerule

\vspace{2ex}
\noindent\spacedlowsmallcaps{What we're reading}:
\vspace{1ex}


\chapter{Moving to the Midwest, making vinegar with elderflowers}

Two years ago my partner and I moved out to the Midwest to go to Graduate school. We live in Vinland Valley, KS, about thirty minutes from Kansas City and the convergence of the Kaw and Missouri Rivers. This is Kansas, but NE Kansas is a little different than we expected. We live on a wooded hill just south of a small prairie wetland. It's buggy, humid, green, and less flat cornfield (though we do have those) and more gently rolling hills. I'm not going to lie, coming from Spokane, WA we had our concerns about moving out here, to what I pecieved to be the big flat, but having lived here for a few years, we've grown to love it; far from homogenous, the flint hills and river valley of Northeastern Kansas are far from boring, they are beautiful, diverse, and alive places.

Come early summer the ditches by the edge of every dirt road out by our house is filled with a mixed thicket of dogwood, trumpet-flower, sumac, and elderflower. Dog wood is invasive, trumpet-flower is pretty but its also an invasive species. I'll be honest and say I don't know whether sumac or elderflower are invasive but they both make for excellent, tasty foreagable food. Sumac is simple, look it up and you'll find endless articles about how to make lemonade from the ``lemonade plant.'' Careful! We're not talking about poison sumac (hint: poison sumac has white berries) but rather red sumac. Sumac is sour, pungent, and super tasty. Elderflower is familiar to many, and it is, in aroma, the complete opposite, and in the best way possible. It is sweet, heady, and floral, it smells incredible, but unlike the tasty elderberries that form late in the season, transforming this edible flower's delicious aroma into something edible can be tough. Elderflower water is nice in a pinch, but not in the same bursting-with-flavor way as sumac water. So, with the help and experience of our foraging land-lady, we set out to make elderflower champagne. I've tried someone else's, and it was delicious. Unfortunately, ours didn't become sweet and bubbly, we left it out far too long for that. But 18 months in we've been rewarded with something else delicious, a sweet and floral vinegar, perfect for all applications where apple-cider vinegar is called for, but with a complex, floral aroma that just can't be beat. We make quick pickles with this vinegar and eat them all day long, especially in the summer. Plus, this vinegar fix is super healthy, impossible to mess up, and very easy to make. If you've got an elder flower thicket on your land, or on the side of a road. Then this is an easy way to turn a seasonal flower into a delicious year-round fix. No elderflowers, try the recipe with any sweet-smelling edible flower an tell us how it goes. From Kansas with love, here's our instructions on how to make your own elderflower vinegar.

\section*{Dana \& Matt's Elderflower Vinegar Recipe}

\setlist[description]{labelindent=2em}
\begin{description}
\item [1] Find a thicket of elderflower, elderflowers love moisture, and thrive in marginal ecosystems, look in ditches along the edges of prairie or woods. A lot of land where we leave was pastured by hedge systems in the early twentieth century, the ditches alongside these fields, where restored prairie meets oak and hedge-tree is a great spot to look for this marginal growth.
\item [2] Once you find a thicket, look for blooms that are already open, you'll know their good to pick if they have a powerful smell, no smell? Don't pick them, wait a little longer. To harvest, clip below the first division of the branching flower-head so that the entire head of tiny flower holds together. If you want to harvest elderberries later on the same season, be sure to leave some heads in tact. No flower left behind means no berries later on.
\item [3] When you get home DO NOT clean off the elderflowers. Especially if it's dusty out, rinsing the flowers is a tempting tactic but dust and dirt will settle off in the sediment of the vinegar ferment, but if you rinse them that great floral aroma will be gone forever.
\item [4] Make a solution of water, lemon (or other acid), and water (about 1 cup sugar to 1 tbsp lemon juice to 1 gallon of water). You can adjust to your preference, but this is a good, inexpensive starting place. The solution should taste sweet, not particularly sour. If it doesn't taste sweet, sorry about my tendency to use heaping cup-fulls and add a little extra sugar.
\item [5] Submerge as many elderflowers as you can fit in your container without over flowing or compacting the flower heads unnecessarily. My first batch was made in 5-gallon bucket, with about three gallons of solution, 1 gallon of head-space, and 1 gallon taken up with elderflowers. Be sure to have a plan to separate the elderflowers from the solution. I use a large mesh bag you can buy at your local homebrew shop? Or you can ferment in a cooler, or bucket with a spigot at the bottom so that you can draw off the liquid and leave behind the flowers.
\item [6] After fermentation begins (look for a bubbles, a pungent hopefully pleasant but sometimes funky smell) then remove the flowers. At this juncture, DO NOT taste. While vinegar or champagne is safe to drink, until the solution becomes adequately acidic and/or alcoholic, all sorts of unfriendly bacteria could be growing in the solution. You can search for safe ph and test that way. But if I'm making vinegar I take a different tactic, time.
\item [7] Sometime between a week and a year (who knows, every culture is different) your solution will start to smell both acidic AND tasty (don't be discourage if it doesn't smell tasty for a while). Bacteria are magical and somehow it almost always ends up tasty in the end. You'll probably also be able to see the ‘mother' culture form in a disk near the top of your ferment. Together, these are strong signs that everything is working good. PH is a safe way to determine food safety, but this is my test: if it smells good and its been at least a month (so fermentation had time to begin) then I'll happily take a sip.
\item [8] Enjoy your vinegar. Put it in a bottle. Give it to a friend. If it tastes extra-awesome, save it and use it to help kick start another vinegar batch.

\end{description}

\chapter{the wind is enough to compete with}


\chapter{Recipe: Spiced Pear Bread}


\chapter{How's your garden growing?}


\newpage
\thispagestyle{empty}
\newgeometry{left=1cm, right=1cm, top=4cm}

\begin{center}
\begin{minipage}{0.6\textwidth}
\spacedlowsmallcaps{A note on the title:}

\vspace{1em}
\noindent The balsamroot is a genus of flowering plants which grow profusely on the open hillsides of Spokane and the West. Its deep taproot allows it to thrive in arid conditions and the entire plant is edible with many medicinal uses. Its presence is a signifier of the health of the local ecosystem.
\end{minipage}
\end{center}

\end{document}
